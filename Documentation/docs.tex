% !TEX encoding = UTF-8 Unicode
\documentclass[14pt,a4paper]{article}

\usepackage{lipsum}
\usepackage[margin=2cm,includefoot]{geometry}
\usepackage[utf8]{inputenc}
\usepackage[russian]{babel}
\usepackage{amssymb}
\usepackage{amsmath}
\usepackage{amsthm}
\usepackage{latexsym}
\usepackage{dsfont}
\usepackage[linesnumbered]{algorithm2e}
\usepackage{mathtools}

\usepackage{tikz} 
\usepackage{tikz-qtree}

\usepackage{graphicx}
\usepackage{float}
\graphicspath{ {./} }

\usepackage{tabularx}
\usepackage{makecell}

\usepackage{titling}
\renewcommand\maketitlehooka{\null\mbox{}\vfill}
\renewcommand\maketitlehookd{\vfill\null}

%\DeclarePairedDelimiter\ceil{\lceil}{\rceil}
%\DeclarePairedDelimiter\floor{\lfloor}{\rfloor}

%\def\changemargin#1#2{\list{}{\rightmargin#2\leftmargin#1}\item[]}
%\let\endchangemargin=\endlist 

%\newcommand\tab[1][0.7cm]{\hspace*{#1}}

%\newcommand{\hm}[1]{#1\nobreak\discretionary{}{\hbox{\ensuremath{#1}}}{}}

%\renewcommand*{\qed}{\hfill\ensuremath{\blacksquare}}%

\usepackage{fancyhdr}
\pagestyle{fancy}
\thispagestyle{empty}
\fancyhead{}
\fancyfoot{}
\fancyfoot[R]{ \thepage\ }
\renewcommand{\headrulewidth}{0pt}
\renewcommand{\footrulewidth}{1pt}

\usepackage{hyperref}
\hypersetup{
    colorlinks=false, %set true if you want colored links
    linktoc=all,     %set to all if you want both sections and subsections linked
    %linkcolor=blue,  %choose some color if you want links to stand out
}

\begin{document}

\text{}
\vskip 8cm
\begin{center}
\begin{minipage}{0.8\textwidth}
\begin{center}
\Huge Веб-приложение для распознавания музыкальных инструментов
\end{center}
\end{minipage}
\end{center}

\vskip 7cm

\begin{flushright}
\Large \underline{Авторы}: \\
Бухараев Алим \\
Прохоров Юрий \\
Савелов Михаил \\
Яушев Фарух
\end{flushright}

\vskip 3.6cm

\begin{center}
2019
\end{center}

\newpage

\tableofcontents

\newpage 

\section{Описание задачи}

\subsection{Аннотация}
\subsection{Техническое задание}
\subsection{Используемые средства}

\newpage

\section{Методика}
\subsection{Обработка цифровых сигналов}
\subsection{Подготовка обучаемой выборки}
\subsection{Модель распознавания}
\subsection{Реализация пользовательского интерфейса}
\subsection{Серверная часть}

\newpage

\section{Документация}
\subsection{Описание файлов}
\subsection{Конкретные файлы}

\newpage

\section{Ссылки}

\end{document}



